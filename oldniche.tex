
\documentstyle[12pt]{article}
\title{Tournament Selection and the Loss of Genetic Diversity}
\author{David E. Goldberg\\S. J. Chang\\C. K. Oei}
\begin{document}
\maketitle
\input epsf

{\noindent\narrower
Abstract:
In this article we examine tournament selection when used in combination with
several
mechanisms for preserving genetic diversity.  We make analytical predictions
for the rate of loss of genetic diversity for sharing and Boltzman tournament
selection, and propose a new method.  We test our predictions using computer
simulations.
}

\section{Introduction}
Tournament selection has several advantages over other methods of selection:
low noise, local interactions, and rank-based selection.  Unfortunately,
attempts at preserving genetic diversity through sharing and tournament
selection have been unsatisfactory because they require a large population to
sustain the niches.

\section{Tournament Selectiona and Sharing}
To see why tournament selection runs into problems when combined with niching,
let us consider the following simple model.  Consider a population of
single-bit strings, each with equal fitness.  Sharing makes the sub-population
with the least number of members the most fit, after the sharing is taken into
account.  Let $n$ be the number of 1's in the population, which is of size $N$.
Then the time evolution of the system is given by:
\begin{equation}
n_+=\left\{ \begin{array}{ll}
	\frac{n^2}{N} & \mbox{for $n>N/2$} \\
	\frac{n^2}{N} + 2 \frac{(N-n)n}{N} & \mbox{for $n<N/2$}
	\end{array}
\right. 
\end{equation}
where $n_+$ is the number of 1's in the next generation.  It is easier to
see what is happening if we look at the ratio of the number of 1's to the
total population: $r=n/N$.  The evolution becomes:
\begin{equation}
r_+=\left\{ \begin{array}{ll}
	r^2 & \mbox{for $r>1/2$}\\
	-r^2+2r & \mbox{for $r<1/2$}
	\end{array}
\right.
\end{equation}
If sharing works, then $r=1/2$ should be a stable fixed point of this mapping.
Notice, however, that the mapping is actually discontinuous at $r=1/2$.
If we start with almost the same number of 0's as 1's, with slightly more 1's,
then in a single generation the ratio would jump to 3:1 in favor of the 0's.
The method of sharing overcompensates and overshoots the equilibrium.

The system has two unstable fixed points of order 1 at 0 and 1, and two unstable
fixed points of order 2 at $\frac{3-5^{1/2}}{2}$ and $\frac{5^{1/2}-1}{2}$.
In fact, since the slope of the mapping is always greater than 1, there are
no stable fixed points anywhere and thus the trajectory of the system is
always aperiodic.  We verified numerically that the system is in fact chaotic
with a Lyapunov exponent of about $0.21$.

If we think of the system in terms of control theory, then what we are trying
to do is to control a system that would, if left by itself, drift away
from the desired equilibrium.  In this case, our controlling force is too
strong, and the 1-generation time lag between the measurement of the system
and the application of the control causes the controlled system to overshoot
the equilibrium, causing feedback and chaos.  We will use this point of view
to correct the situation later in this paper.

Actually, the fact that the system is chaotic does not really tell us that
sharing does not work.  In order to show that sharing fails we must estimate
the rate of loss of genetic diversity, and to do this, we will use a more
involved model.  Instead of thinking of two sub-populations, we will consider
a system with many sub-populations and assume that what happens in one
sub-population does not directly affect what happens in another population.
This model is similar to the mean-field theory that is often used to describe
the behavior of physical systems.  Let us assume we have a large population
of size $N$ with $k_0$ sub-populations at time $t=0$.  
Again, we will take all the fitness values to be equal for now.

Let $p(k,x,t)$ be the number of individuals at
time $t$ in the $x$-th niche, where the
niches are sorted according to their sizes and $k=k(t)$ is the number of
niches remaining at time $t$.  The largest niche will have
$p(k,k,t)$ members and the smallest niche will have $p(k,1,t)$ members.  What we
expect is that for large $k$, the system will quickly collapse onto
a slow manifold that does not depend on the initial conditions except for
the value of $N$ and $k_0$.
Let us work in the rescaled population distribution:
\begin{equation}
F(x/k,t)=\frac{k}{N} p(k,x,t)
\end{equation}
What we are guessing is that as time progresses, the system loses more and
more niches, but the rescaled population distribution remains unchanged.
If such a quasi-steady-state solution exists, it is given by:
\begin{equation}
F^*(y)=\lim_{t \rightarrow \infty} F(y,t)
\end{equation}
So if our quasi-steady-state assumption holds, then
when we have a large number of niches ($k \gg 1$)
and evolution has taken place for a while ($t \gg 1$), 
$p(k,x,t)$ follows a simple scaling law:
\begin{equation}
p(k,x,t) \approx \frac{N}{k} F^*(x/k)
\end{equation}
We verified this scaling law numerically.  It turns out that it is possible
to solve for this quasi-steady-state solution, $F^*$ analytically.
The key is to postulate the existance of a solution $F^*(y)$
that is symmetric about $y=1/2$ after the tournament reproduction phase; that is, the
most endangered niches remain highly endangered, the most populated
niches become endangered, and the niches in between the extremes
remain about the same.  Numerical studies pointed out this symmetry.
Using these assumptions, we can write the time evolution of the system
as follows:

Since all the fitnesses are equal, the outcome of the tournaments depends
only on how many individuals are in the niches.  Let $F_t$ be the
rescaled population distribution after the tournament phase.  Since
the niches are ranked according to their sizes, we can take the expectation:
\begin{equation}
F_t(x,t)=2\int_x^1{dy\,F(x,t) F(y,t)}
\end{equation}

After the tournament phase, some of the niches that were heavily
populated are no longer heavily populated, and some of the niches that
were endangered are no longer endangered.  This means that we must resort
the niches according to their new sizes.  Ordinarily, this would be
extremely difficult to do analytically, but
because the solution is symmetric by assumption, the sorting phase is
given by:
\begin{equation}
F_{s}(x,t)=F_t(x/2,t)
\end{equation}

Now we just renormalize the probability distribution so that the integral
is unity:
\begin{equation}
F(x,t+1)=C F_s(x,t)
\end{equation}

We can write the time evolution of the system as an operator equation:
\begin{equation}
F(x,t+1)=\Psi[F(x,t)]
\end{equation}
where the operator $\Psi$ is given by:
\begin{equation}
\Psi[w(x)]=\Phi[w(x)]/\int_0^1{dx\, \Phi[w(x)]}
\end{equation}
\begin{equation}
\Phi[w(x)]=w(x/2)\int_{x/2}^1{dy\,w(y)}
\end{equation}
For the quasi-steady-state solution, we must solve:
\begin{equation}
F^*(x)=\Psi[F^*(x)]
\end{equation}
These equations are straightforward to solve, yielding the quasi-steady-state
solution:
\begin{equation}
F^*(x)=\frac{\pi}{2}\sin(\frac{\pi x}{2})
\end{equation}
It is straightforward to verify that after the tournament phase, this
solution is symmetric about $x=1/2$, and so we have a self-consistent
solution.  The fact that numerical experiments agree with this analysis
indicates that this is indeed a stable solution.

\begin{figure}
\epsffile{predpop.eps}
\caption{Continuum and Mean Field Comparisons}
\end{figure}

\begin{figure}
\epsffile{empvscon.eps}
\caption{Continuum and Empirical Comparisons}
\end{figure}

Now that we have the distribution of the population, it is straightforward
to compute the probability that a certain sub-population will lose every
tournament that it enters, in the mean-field approximation.  This will
tell us how the number of niches, $k$, depends on time.
The rate of loss is
\begin{equation}
\frac{\partial k}{\partial t}= -
\int_0^k{dx\,(\int_0^{x/k}\frac{\pi}{2}\sin(\frac{\pi}{2}y)dy)^{
	\frac{\pi N}{2 k} 2 \sin(\frac{\pi x}{2 k}}}
\end{equation}
The factor of two in the exponent comes because each individual enters
twice in tournaments.

A little massaging gives:
\begin{equation}
\frac{\partial k}{\partial t}= -
k \int_0^1{dx\,(1-\cos(\frac{\pi x}{2}))^{\frac{\pi N}{k} \sin(\frac{\pi x}{2})}}
\end{equation}
or:
\begin{equation}
\frac{\partial k}{\partial t}=-k G(N/k)
\end{equation}
where $G$ is determined by the previous equation.
It can be shown that only the values of the integrand near the endpoint at $x=1$
matters in the integral when $N/k$ is large.  Doing the asymptotic
expansion yields:

\begin{equation}
\frac{\partial k}{\partial t}=-\frac{2}{\pi^2}\frac{k^2}{N}
\end{equation}

Using the initial condition $k(0)=k_0$ we get:
\begin{equation}
k=\frac{1}{\frac{1}{k_0}+\frac{2t}{\pi^2 N}}
\end{equation}

The following are two numerical studies to check this result.  We used
a fixed number of individuals per niche, varied the number of niches in
the initial population, and asked how many niches survive fifty generations.
\begin{figure}
\epsffile{survniche.eps}
\caption{Number of niches that survive 50 generations vs. initial number of niches}
\end{figure}
We also took a single run and asked how many niches survived as a function
of time.
\begin{figure}
\epsffile{survnic2.eps}
\end{figure}

A simple rule of thumb for computing the required population size for 
supporting $k_0$ niches is to set $\frac{\partial k}{\partial t} \ll 1$
at $t=0$, which gives:
\begin{equation}
N \gg k_0^2
\end{equation}
The number of niches we can support for a fixed number of generations
scales as the square root of the population size.  Thus, if we want
to support twice as many niches, we need four times as many people in
the population.

\section{Boltzman Tournament Selection}
To analyze Boltzman tournament selection, we will again assume that
all the individuals have the same fitness.  This corresponds to the
high-temperature limit.  The true behavior of the system will be somewhat
worse than our prediction, depending on the temperature used in the
algorithm; the finite temperature and slight differences in fitness
will cause a bias in the tournaments that will kill off the worst niches
faster than we predicted.  However, our estimate should yield an estimate
of the best that we can expect.  What we will find is the opposite of
what we found before.  In the sharing method, the problem was that the
controlling force was so strong that it overshot the equilibrium.  In
Boltzman tournament selection, there is no controlling force and the
system will drift away from the equilibrium in a manner similar to
a random walk.   Again, we will assume that the actions of a particular
niche will not affect the actions of the other niches directly.

Rather than concentrate on the entire population as we did before, we
will concentrate on a single niche.  We will assume that at $t=0$,
all niches have the same number of individuals.  Now 
we can use generating functions to do the analysis.
Let $l$ represent the state with a single individual; $l^2$ represents
the state with two individuals; $\frac{1}{2}(1+l)$ represents the
state which has fifty percent probability of having no individuals, and
fifty precent probability of having one individual.  Using this notation,
we can write the time evolution of the niche in the mean-field approximation.
At $t=0$, the niche has $n$ individuals, and so the initial state is
represented by the function $l^n$.  Each successive generation, the
individuals in the niche can win both tournaments, win one tournament,
or lose both tournaments, with probabilities $1/4$, $1/2$, and $1/4$,
respectively.  We represent this by substituting $l$ by 
$\frac{1}{4}l^2+\frac{1}{2}l+\frac{1}{4}=(\frac{l+1}{2})^2$ each generation.
The result is an iterative equation:
\begin{eqnarray}
g(l,0)&=&l^n \\
g(l,t)&=&g((\frac{l+1}{2})^2,t-1)
\end{eqnarray}
Note that $g(0,t)$ gives the probability that the niche will have no
individuals at time $t$.  Our estimate for the number of niches remaining
at a given time is therefore:
\begin{equation}
k(t)=k_0 (1-g(0,t))
\end{equation}
We can get a better idea of how the system behaves by using asymptotics
on the iterative equation $l(t+1)= (\frac{l(t)+1}{2})^2$.
Let $\epsilon(t)=1-l(t)$.  The iterative equation becomes:
\begin{equation}
\epsilon(t+1)=\epsilon(t)(1-\frac{1}{4}\epsilon(t))
\end{equation}
For large $t$, the behavior is given by:
\begin{equation}
\epsilon(t)=4/t+\ldots
\end{equation}
Plugging this back into the definition of the generating function gives:
\begin{equation}
g(0,t)=1-4/t+\ldots
\end{equation}
And so the number of niches alive at time $t$ for large $t$ is
\begin{equation}
k(t)=\frac{4 k_0}{t}+\ldots
\end{equation}
Again, we see that the loss of niches is a severe problem.
\begin{figure}
\epsffile{bolztourn.eps}
\caption{500 niches, 10 people/niche}
\end{figure}

\begin{figure}
\epsffile{bolztourn2.eps}
\caption{500 niches, 17 people/niche}
\end{figure}

\section{Continuous Update Tournament Selection}
One way of repairing the sharing method to work with tournament selection
is to refine the controlling force.  Before, we tried to create the
new generation by looking only at how many individuals existed in that
niche in the old generation.  What we will try to do now is to create
the new generation by looking at how many individuals exist in the new
generation as it is being created.  That is, when two individuals
compete, we look in the new generation and see which one has fewer
individuals in its niche; this individual wins the tournament.  When we
tested this result numerically, we found that with ten individuals per
niche, we could support 500 niches for 50 generations with no losses.
Niches still can be lost, however.  Consider the case where there are three
niches, A, B, and C, with only one individual in the old generation.
Let us also assume that the shuffling was so unlucky as to cause these niches
to compete only with each other.  Since all three niches are in identical
predicaments, we consider only the case where A fights B in the first
tournament and loses.  In the second tournament, if A fights C and loses
again, it is eliminated.  Otherwise, it will be have at least one copy
in the next generation.

We can analyze the quasi-steady-state of this algorithm in the following way.
Let us look at the
size of the niche in the new generation during the reproduction.  We will
assume that there are already a substantial number of individuals in
the target population.  Under these conditions, the algorithm will have
the same noise characteristics as another algorithm: simply take two
individuals at random in the population and replace the more redundant
individual with another copy of the less redundant individual.
Let $w(x)$ be the probability of finding a niche with $N/k-x$ individuals.
Then the rate of production of niches with $N/k-x$ individuals is
\begin{equation}
w(x-1)(w(x-1)+w(x)+w(x+1)+w(x+2)+\ldots)
\end{equation}
and the rate of destruction of niches with $N/k-x$ individuals is $w(x)$.
The quasi-steady-state condition gives that the rate of production equals
the rate of destruction:
\begin{equation}
w(x)=w(x-1)(w(x-1)+w(x)+w(x+1)+w(x+2)+\ldots)
\end{equation}
We will assume that $w(x)$ decreases rapidly in $x$ for large $x$ so that
we can approximate the sum by the first term.
\begin{equation}
w(x) \approx w(x-1)^2
\end{equation}
The general solution to this equation is
\begin{equation}
w(x) \approx C_1 e^{-C_2 2^x}
\end{equation}
where $C_1$ and $C_2$ are arbitrary constants.  We see that $w(x)$ does indeed
decreases rapidly in $x$ for large $x$ so our ansatz is self-consistant.
From this argument, we conclude that large fluctuations from the equilibrium
positition of $N/k$ individuals per niche are extremely unlikely; the
probability of a fluctuation of size $x$ drops as an exponential of an
exponential.  The rate of loss of niches is proportional to the probability
of seeing a fluctuation of size $N/k$:
\begin{equation}
\frac{\partial n}{\partial t} \approx -C_1 k e^{-C_2 2^{N/k}}
\end{equation}
So we see that increasing the niche size by even a single individual
represents a dramatic improvement in the ability of the algorithm to
sustain niches. To insure that we maintain genetic diversity, we should
set $\frac{\partial n}{\partial t} \ll 1$ and solve for $N$.  This
gives
\begin{equation}
N=D k log(E log(F k))
\end{equation}
where $D$, $E$, and $F$ are constants.  We will assume that for any
reasonable value of $k$, the double logarithm is of order unity.  This
gives
\begin{equation}
N=D k
\end{equation}
Using a population size of ten times the number of niches seems to yield
no loss of genetic diversity for any reasonable value of $k$.  Thus, the
number of niches we can support goes up almost linearly in the population
size.

There are several ways to implement this method in a real application.
One way would be to use Goldberg's idea of adjusting the fitness function
with a sharing function and compute the sharing function by using the
target population.  Another method would be to introduce a niche size parameter
$n^*$.  With this method, we would use Goldberg's method of determining
the number of individuals in a niche, but we would determine which individual
wins a tournament in the following manner: if the two niches that the
individuals belong to both have less than $n^*$ members, then the individual
with the better fitness value wins; otherwise the more endangered
individual wins.  The former method would evolve to a population that
has more individuals in the niches that have higher fitness peaks, while the
latter method would evolve to a population that would have about $n^*$
individuals in each of the best $N/n^*$ niches.

We can make a modification of this algorithm to speed it up.  Currently,
each generation requires that we do $O(N^2)$ operations to compute the
number of individuals in the same niche.  Actually, we do not need to
sample the entire target population to get a good estimate of the number
of individuals in a niche.  We only need to look at the last $A k$
individuals in the target population, where $A\approx 5$ for a good
sampling.  This reduces the computational complexity to $O(N k)$ per
generation.

\section{Conclusion}
We have shown that two methods for preserving genetic diversity,
tournament selection with sharing and Boltzman tournament selection,
have serious limitations on the number of niches they can support.
We have also shown that a modification of the tournment selection with
sharing method yields good results.
\end{document}

